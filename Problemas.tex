% ****************************************************************************************
% ************************      NAME OF DOCUMENT      ************************************
% ****************************************************************************************

% =======================================================
% =======         HEADER FOR DOCUMENT        ============
% =======================================================
    % *********   DOCUMENT ITSELF   **************
    \documentclass[12pt, fleqn]{article}                             %Type of docuemtn and size of font and left eq
    \usepackage[margin=1.2in]{geometry}                             %Margins and Geometry pacakge
    \usepackage{ifthen}                                             %Allow simple programming
    \usepackage{hyperref}                                           %Create MetaData for a PDF and LINKS!
    \hypersetup{pageanchor=false}                                   %Solve 'double page 1' warnings in build
    \setlength{\parindent}{0pt}                                     %Eliminate ugly indentation
    \author{Oscar Andrés Rosas}                                     %Who I am

    % *********   LANGUAJE AND UFT-8   *********
    \usepackage[spanish]{babel}                                     %Please use spanish
    \usepackage[utf8]{inputenc}                                     %Please use spanish - UFT
    \usepackage[T1]{fontenc}                                        %Please use spanish
    \usepackage{textcmds}                                           %Allow us to use quoutes
    \usepackage{changepage}                                         %Allow us to use identate paragraphs
    \usepackage{lipsum}                                             %Allow to put dummy text

    % *********   MATH AND HIS STYLE  *********
    \usepackage{ntheorem, amsmath, amssymb, amsfonts}               %All fucking math, I want all!
    \usepackage{mathrsfs, mathtools, empheq}                        %All fucking math, I want all!
    \usepackage{centernot}                                          %Allow me to negate a symbol
    \decimalpoint                                                   %Use decimal point

    % *********   GRAPHICS AND IMAGES *********
    \usepackage{graphicx}                                           %Allow to create graphics
    \usepackage{wrapfig}                                            %Allow to create images
    \graphicspath{ {Graphics/} }                                    %Where are the images :D

    % *********   LISTS AND TABLES ***********
    \usepackage{listings}                                           %We will be using code here
    \usepackage[inline]{enumitem}                                   %We will need to enumarate
    \usepackage{tasks}                                              %Horizontal lists
    \usepackage{longtable}                                          %Lets make tables awesome
    \usepackage{booktabs}                                           %Lets make tables awesome
    \usepackage{tabularx}                                           %Lets make tables awesome
    \usepackage{multirow}                                           %Lets make tables awesome
    \usepackage{multicol}                                           %Create multicolumns

    % *********   HEADERS AND FOOTERS ********
    \usepackage{fancyhdr}                                           %Lets make awesome headers/footers
    \pagestyle{fancy}                                               %Lets make awesome headers/footers
    \setlength{\headheight}{16pt}                                   %Top line
    \setlength{\parskip}{0.5em}                                     %Top line
    \renewcommand{\footrulewidth}{0.5pt}                            %Bottom line

    \lhead{                                                         %Left Header
        \hyperlink{section.\arabic{section}}                        %Make a link to the current chapter
        {\normalsize{\textsc{\nouppercase{\leftmark}}}}             %And fot it put the name
    }

    \rhead{                                                         %Right Header
        \hyperlink{section.\arabic{section}.\arabic{subsection}}    %Make a link to the current chapter
            {\footnotesize{\textsc{\nouppercase{\rightmark}}}}      %And fot it put the name
    }

    \rfoot{\textsc{\small{\hyperref[sec:Index]{Ve al Índice}}}}     %This will always be a footer  

    \fancyfoot[L]{                                                  %Algoritm for a changing footer
        \footnotesize{Grupo 4098}
    }
    
    
    
% ========================================
% ===========   COMMANDS    ==============
% ========================================

    % =====  GENERAL TEXT  ==========
    \newcommand \Quote {\qq}                                        %Use: \Quote to use quotes
    \newcommand \Over {\overline}                                   %Use: \Bar to use just for short
    \newcommand \ForceNewLine {$\Space$\\}                          %Use it in theorems for example
    
    \newenvironment{Indentation}[1][0.75em]                         %Use: \begin{Inde...}[Num]...\end{Inde...}
    {\begin{adjustwidth}{#1}{}}                                     %If you dont put nothing i will use 0.75 em
    {\end{adjustwidth}}                                             %This indentate a paragraph
    \newenvironment{SmallIndentation}[1][0.75em]                    %Use: The same that we upper one, just 
    {\begin{adjustwidth}{#1}{}\begin{footnotesize}}                 %footnotesize size of letter by default
    {\end{footnotesize}\end{adjustwidth}}                           %that's it
        
    % =====  GENERAL MATH  ==========
    \DeclareMathOperator \Space {\quad}                             %Use: \Space for a cool mega space
    \DeclareMathOperator \MiniSpace {\;}                            %Use: \Space for a cool mini space
    \newcommand \Such {\MiniSpace|\MiniSpace}                       %Use: \Such like in sets
    \newcommand \Also {\Space \text{y} \Space}                      %Use: \Also so it's look cool
    \newcommand \Remember[1]{\Space\text{\scriptsize{#1}}}          %Use: \Remember so it's look cool

    \newtheorem{Theorem}{Teorema}[section]                          %Use: \begin{Theorem}[Name]\label{Nombre}...
    \newtheorem{Corollary}{Colorario}[Theorem]                      %Use: \begin{Corollary}[Name]\label{Nombre}...
    \newtheorem{Lemma}[Theorem]{Lemma}                              %Use: \begin{Lemma}[Name]\label{Nombre}...
    \newtheorem{Definition}{Definición}[section]                    %Use: \begin{Definition}[Name]\label{Nombre}...

    \newcommand{\Set}[1]{\left\{ \MiniSpace #1 \MiniSpace \right\}} %Use: \Set {Info}
    \newcommand{\Brackets}[1]{\left[ #1 \right]}                    %Use: \Brackets {Info} 
    \newcommand{\Wrap}[1]{\left( #1 \right)}                        %Use: \Wrap {Info} 
    \newcommand{\pfrac}[2]{\Wrap{\dfrac{#1}{#2}}}                   %Use: Put fractions in parentesis

    \newenvironment{MultiLineEquation}[1]                           %Use: To create MultiLine equations
        {\begin{equation}\begin{alignedat}{#1}}                     %Use: \begin{Multi..}{Num. de Columnas}
        {\end{alignedat}\end{equation}}                             %And.. that's it!
    \newenvironment{MultiLineEquation*}[1]                          %Use: To create MultiLine equations
        {\begin{equation*}\begin{alignedat}{#1}}                    %Use: \begin{Multi..}{Num. de Columnas}
        {\end{alignedat}\end{equation*}}                            %And.. that's it!


    % =====  LOGIC  ==================
    \DeclareMathOperator \doublearrow {\leftrightarrow}             %Use: \doublearrow for a double arrow
    \newcommand \lequal {\MiniSpace \Leftrightarrow \MiniSpace}     %Use: \lequal for a double arrow
    \newcommand \linfire {\MiniSpace \Rightarrow \MiniSpace}        %Use: \lequal for a double arrow
    \newcommand \longto {\longrightarrow}                           %Use: \longto for a long arrow

    % =====  NUMBER THEORY  ==========
    \DeclareMathOperator \Naturals  {\mathbb{N}}                     %Use: \Naturals por Notation
    \DeclareMathOperator \Primes    {\mathbb{P}}                     %Use: \Naturals por Notation
    \DeclareMathOperator \Integers  {\mathbb{Z}}                     %Use: \Integers por Notation
    \DeclareMathOperator \Racionals {\mathbb{Q}}                     %Use: \Racionals por Notation
    \DeclareMathOperator \Reals     {\mathbb{R}}                     %Use: \Reals por Notation
    \DeclareMathOperator \Complexs  {\mathbb{C}}                     %Use: \Complex por Notation

    % === LINEAL ALGEBRA & VECTORS ===
    \DeclareMathOperator \LinealTransformation {\mathcal{T}}        %Use: \LinealTransformation for a cool T
    \newcommand{\Mag}[1]{\left| #1 \right|}                         %Use: \Mag {Info} 

    \newcommand{\pVector}[1]{                                       %Use: \pVector {Matrix Notation} use parentesis
        \ensuremath{\begin{pmatrix}#1\end{pmatrix}}                 %Example: \pVector{a\\b\\c} or \pVector{a&b&c} 
    }
    \newcommand{\lVector}[1]{                                       %Use: \lVector {Matrix Notation} use a abs 
        \ensuremath{\begin{vmatrix}#1\end{vmatrix}}                 %Example: \lVector{a\\b\\c} or \lVector{a&b&c} 
    }
    \newcommand{\bVector}[1]{                                       %Use: \bVector {Matrix Notation} use a brackets 
        \ensuremath{\begin{bmatrix}#1\end{bmatrix}}                 %Example: \bVector{a\\b\\c} or \bVector{a&b&c} 
    }
    \newcommand{\Vector}[1]{                                        %Use: \Vector {Matrix Notation} no parentesis
        \ensuremath{\begin{matrix}#1\end{matrix}}                   %Example: \Vector{a\\b\\c} or \Vector{a&b&c}
    }

    % MATRIX
    \makeatletter                                                   %Example: \begin{matrix}[cc|c]
    \renewcommand*\env@matrix[1][*\c@MaxMatrixCols c] {             %WTF! IS THIS
        \hskip -\arraycolsep                                        %WTF! IS THIS
        \let\@ifnextchar\new@ifnextchar                             %WTF! IS THIS
        \array{#1}                                                  %WTF! IS THIS
    }                                                               %WTF! IS THIS
    \makeatother                                                    %WTF! IS THIS

    % TRIGONOMETRIC FUNCTIONS
    \newcommand{\Cos}[1]{\cos\Wrap{#1}}                             %Simple wrappers
    \newcommand{\Sin}[1]{\sin\Wrap{#1}}                             %Simple wrappers

    % === COMPLEX ANALYSIS ===
    \newcommand \Cis[1]  {\Cos{#1} + i \Sin{#1}}                    %Use: \Cis for cos(x) + i sin(x)
    \newcommand \pCis[1] {\Wrap{\Cis{#1}}}                          %Use: \pCis for the same ut parantesis
    \newcommand \bCis[1] {\Brackets{\Cis{#1}}}                      %Use: \bCis for the same to Brackets




% =====================================================
% ============        COVER PAGE       ================
% =====================================================
\begin{document}
\begin{titlepage}

    \center
    % ============ UNIVERSITY NAME AND DATA =========
    \textsc{\Large Algebra Superior 2}\\[0.5cm] 
    \textsc{\large Grupo 4098}\\[1.0cm]

    % ============ NAME OF THE DOCUMENT  ============
    \rule{\linewidth}{0.5mm} \\[1.0cm]
        { \huge \bfseries Soluciones y Demostraciones}\\[1.0cm] 
    \rule{\linewidth}{0.5mm} \\[1.5cm]
     
    % ============  MY INFORMATION  =================
    \begin{minipage}{0.55\textwidth}
        \begin{flushleft}
            \footnotesize{
            \textbf{\textsc{Alumnos:}}\\
                \begin{itemize}
                    \item Palacios Rodríguez Ricardo Rubén
                    \item Rosas Hernandez Oscar Andres
                    \item José Martín Panting Magaña
                    \item Raúl Leyva Cedillo
                    \item Angel Mariano Guiño Flores
                    \item Gloria Guadalupe Cervantes Vidal
                    \item David Iván Morales Campos
                    \item Aaron Barrera Tellez
                    \item Elias Garcia Alejandro
                    \item Víctor Hugo García Hernández
                    \item Oscar Márquez Esquivel
                \end{itemize}
            }
        \end{flushleft}
    \end{minipage}
    ~
    \begin{minipage}{0.4\textwidth}
        \begin{flushright} \footnotesize
            \textbf{\textsc{Profesor: }}\\
            Leonardo Faustinos Morales
        \end{flushright}
    \end{minipage}\\[3,5cm]

    
    % ====== DATE ================
    {\large 12 de Septiembre de 2017}\\[1cm] 

    \vfill

\end{titlepage}


% =====================================================
% ========                INDICE              =========
% =====================================================
\tableofcontents{}
\label{sec:Index}

\clearpage




% ======================================================================================
% ==================================     DIVISIBILIDAD    ==============================
% ======================================================================================
\section{Divisibilidad}

    \subsection{Problema 1}
    \subsection*{Algoritmo de Euclides: Encontrar el GCD(A, B)}

            Calcular el $GCD(2947, 3997)$
            \begin{itemize}
                \item $(a:2947) = (b:3997)(q:0) + (r:2947)$
                \item $(a:3997) = (b:2947)(q:1) + (r:1050)$
                \item $(a:2947) = (b:1050)(q:2) + (r:847)$
                \item $(a:1050) = (b:847)(q:1) + (r:203)$
                \item $(a:847) = (b:203)(q:4) + (r:35)$
                \item $(a:203) = (b:35)(q:5) + (r:28)$
                \item $(a:35) = (b:28)(q:1) + (r:7)$
                \item $(a:28) = (b:7)(q:4) + (r:0)$
            \end{itemize}                    
            Así que $GCD(2947, 3997) = 7$\\

            Calcular el $GCD(2689, 4001)$
            \begin{itemize}
                \item $(a:2689) = (b:4001)(q:0) + (r:2689)$ 
                \item $(a:4001) = (b:2689)(q:1) + (r:1312)$ 
                \item $(a:2689) = (b:1312)(q:2) + (r:65)$   
                \item $(a:1312) = (b:65)(q:20) + (r:12)$  
                \item $(a:65) = (b:12)(q:5) + (r:5)$ 
                \item $(a:12) = (b:5)(q:2) + (r:2)$
                \item $(a:5) = (b:2)(q:2) + (r:1)$    
                \item $(a:2) = (b:1)(q:2) + (r:0)$   
            
            \end{itemize}                    
            Así que $GCD(2689, 4001) = 1$\\

            \clearpage



            Calcular el $GCD(7469, 2464)$
            \begin{itemize}
                \item $(a:7469) = (b:2464)(q:3) + (r:77)$                       
                \item $(a:2464) = (b:77)(q:32) + (r:0)$
            \end{itemize}
            Así que $GCD(7469, 2464) = 77$\\


            Calcular el $GCD(2947, 3997)$
            \begin{itemize}
                \item $(a:2947) = (b:3997)(q:0) + (r:2947)$
                \item $(a:3997) = (b:2947)(q:1) + (r:1050)$
                \item $(a:2947) = (b:1050)(q:2) + (r:847)$
                \item $(a:1050) = (b:847)(q:1) + (r:203)$
                \item $(a:847) = (b:203)(q:4) + (r:35)$
                \item $(a:203) = (b:35)(q:5) + (r:28)$
                \item $(a:35) = (b:28)(q:1) + (r:7)$
                \item $(a:28) = (b:7)(q:4) + (r:0)$
            \end{itemize}                    
            Así que $GCD(2947, 3997) = 7$\\


            Calcular el $GCD(1109, 4999)$
            \begin{itemize}
                \item $(a:1109) = (b:4999)(q:0) + (r:1109)$
                \item $(a:4999) = (b:1109)(q:4) + (r:563)$
                \item $(a:1109) = (b:563)(q:1) + (r:546)$
                \item $(a:563) = (b:546)(q:1) + (r:17)$
                \item $(a:546) = (b:17)(q:32) + (r:2)$
                \item $(a:17) = (b:2)(q:8) + (r:1)$
                \item $(a:2) = (b:1)(q:2) + (r:0)$
            \end{itemize}                    
            Así que $GCD(1109, 4999) = 1$\\

    \clearpage
    \subsection{Problema 3}
    \subsection*{Algoritmo de Euclides Extendido y Coeficientes de Bezut}

            Encontremos los coeficientes de $243x + 198y = 9$
            \begin{itemize}
                \item $(a:243) = (b:198)(q:1) + (r:45)$
                \item $(a:198) = (b:45)(q:4) + (r:18)$
                \item $(a:45) = (b:18)(q:2) + (r:9)$
                \item $(a:18) = (b:9)(q:2) + (r:0)$ 
            \end{itemize}

            El proceso para encontrar los coeficientes de Bezut son:

            \begin{itemize}
                \item $(a':243) = (a':243)(m:1) + (b':198)(n:0)$
                \item $(b':198) = (a':243)(m:0) + (b':198)(n:1)$
            \end{itemize}

            \begin{itemize}
                \item $(r:45) = (a:243) - (b:198)(1:1)  =  (a':243)(m:1) + (b':198)(n:-1)$
                \item $(r:18) = (a:198) - (b:45)(1:4)  =  (a':243)(m:-4) + (b':198)(n:5)$
                \item $(r:9) = (a:45) - (b:18)(1:2)  =  (a':243)(m:9) + (b':198)(n:-11)$
                \item $(r:0) = (a:18) - (b:9)(1:2)  =  (a':243)(m:-22) + (b':198)(n:27)$
            \end{itemize}

            Por lo tanto el $GCD(243, 198) = 9$\\
            Y los números de Bezut son $(243, 198) = (9, -11)$\\                       
            Y la Identidad de Bezut es: $(GCD:9) = (a':243)(m:9) +(b':198)(n:-11)$\\



            \clearpage


            Encontremos los coeficientes de $71x + 50y = 1$
            \begin{itemize}
                \item $(a:71) = (b:50)(q:1) + (r:21)$
                \item $(a:50) = (b:21)(q:2) + (r:8)$
                \item $(a:21) = (b:8)(q:2) + (r:5)$ 
                \item $(a:8) = (b:5)(q:1) + (r:3)$  
                \item $(a:5) = (b:3)(q:1) + (r:2)$   
                \item $(a:3) = (b:2)(q:1) + (r:1)$   
                \item $(a:2) = (b:1)(q:2) + (r:0)$  
            \end{itemize}

            El proceso para encontrar los coeficientes de Bezut son:

            \begin{itemize}
                \item $(a':71) = (a':71)(m:1) + (b':50)(n:0)$
                \item $(b':50) = (a':71)(m:0) + (b':50)(n:1)$    
            \end{itemize}

            \begin{itemize}
                \item $(r:21) = (a:71) - (b:50)(1:1)  =  (a':71)(m:1) + (b':50)(n:-1)$
                \item $(r:8) = (a:50) - (b:21)(1:2)  =  (a':71)(m:-2) + (b':50)(n:3) $
                \item $(r:5) = (a:21) - (b:8)(1:2)  =  (a':71)(m:5) + (b':50)(n:-7)  $
                \item $(r:3) = (a:8) - (b:5)(1:1)  =  (a':71)(m:-7) + (b':50)(n:10)  $
                \item $(r:2) = (a:5) - (b:3)(1:1)  =  (a':71)(m:12) + (b':50)(n:-17) $
                \item $(r:1) = (a:3) - (b:2)(1:1)  =  (a':71)(m:-19) + (b':50)(n:27) $
                \item $(r:0) = (a:2) - (b:1)(1:2)  =  (a':71)(m:50) + (b':50)(n:-71) $ 
            \end{itemize}

            Por lo tanto el $GCD(71, 50) = 1$\\
            Y los números de Bezut son $(71, 50) = (-19, 27)$\\                       
            Y la Identidad de Bezut es: $(GCD:9) = (GCD:1) = (a':71)(m:-19) +(b':50)(n:27)$\\



            \clearpage



            Encontremos los coeficientes de $43 + 64 = 1$
            \begin{itemize}
                \item $(a:43) = (b:64)(q:0) + (r:43)$ 
                \item $(a:64) = (b:43)(q:1) + (r:21)$ 
                \item $(a:43) = (b:21)(q:2) + (r:1) $ 
                \item $(a:21) = (b:1)(q:21) + (r:0) $ 
            \end{itemize}

            El proceso para encontrar los coeficientes de Bezut son:

            \begin{itemize}
                \item $(a':43) = (a':43)(m:1) + (b':64)(n:0)$
                \item $(b':64) = (a':43)(m:0) + (b':64)(n:1)$   
            \end{itemize}

            \begin{itemize}
                \item $(r:43) = (a:43) - (b:64)(1:0)  =  (a':43)(m:1) + (b':64)(n:0)  $ 
                \item $(r:21) = (a:64) - (b:43)(1:1)  =  (a':43)(m:-1) + (b':64)(n:1) $ 
                \item $(r:1) = (a:43) - (b:21)(1:2)  =  (a':43)(m:3) + (b':64)(n:-2)  $ 
                \item $(r:0) = (a:21) - (b:1)(1:21)  =  (a':43)(m:-64) + (b':64)(n:43)$  
            \end{itemize}

            Por lo tanto el $GCD(43, 64) = 1$\\
            Y los números de Bezut son $(43, 64) = (3, -2)$\\                       
            Y la Identidad de Bezut es: $(GCD:1) = (a':43)(m:3) +(b':64)(n:-2)$\\




            \clearpage



            Encontremos los coeficientes de $93 + 81 = 3$
            \begin{itemize}
                \item $(a:93) = (b:81)(q:1) + (r:12)$
                \item $(a:81) = (b:12)(q:6) + (r:9)$
                \item $(a:12) = (b:9)(q:1) + (r:3)$
                \item $(a:9) = (b:3)(q:3) + (r:0)$     
            \end{itemize}

            El proceso para encontrar los coeficientes de Bezut son:

            \begin{itemize}
                \item $(a':93) = (a':93)(m:1) + (b':81)(n:0)$
                \item $(b':81) = (a':93)(m:0) + (b':81)(n:1)$ 
            \end{itemize}

            \begin{itemize}
                \item $(r:12) = (a:93) - (b:81)(1:1)  =  (a':93)(m:1) + (b':81)(n:-1)$
                \item $(r:9) = (a:81) - (b:12)(1:6)  =  (a':93)(m:-6) + (b':81)(n:7) $
                \item $(r:3) = (a:12) - (b:9)(1:1)  =  (a':93)(m:7) + (b':81)(n:-8)  $
                \item $(r:0) = (a:9) - (b:3)(1:3)  =  (a':93)(m:-27) + (b':81)(n:31) $
            \end{itemize}

            Por lo tanto el $GCD(93, 81) = 3$\\
            Y los números de Bezut son $(93, 81) = (7, -8)$\\                       
            Y la Identidad de Bezut es: $(GCD:3) = (a':93)(m:7) +(b':81)(n:-8)$\\




            Encontremos los coeficientes de $10x + 15y = 5$
            ... Espera, este es muy obvio, es simplemente
            $(GCD:5) = (a':10)(m:-1) +(b':15)(n:1)$

            Mientras que el de $6x + 5y = 1$ es $(GCD:1) = (a':6)(m:1) +(b':5)(n:-1)$

            Por lo tanto:
            $(GCD:1) = (a':6)(m:1) +(b':10)(n:1) + (c': 15)(o:-1)$


    \clearpage
    \subsection{Problema 5}
    \subsection*{¿Cuantos enteros hay entre 100 y 1000 que sean divisibles entre 7?}

        Empecemos porque el primero es 105, de ahi hay 127 más, pues 105 + (127 * 7) = 994.

        Por lo tanto son 128 enteros.

        Otro truco es aplicar el algoritmo de la división y ver que 
        $1000 = 7(142) + 6$ y $100 = 7(14) + 2$ y $142-14 = 128$.


    \subsection{Problema 7}
    \subsection*{Mostrar 3 enteros que son relativos, pero no primos relativos a pares}

        Esto simplemente no se puede, si un conjunto es primo relativo, entonces lo será cada
        par de sus elementos.


    \subsection{Problema 9}
    \subsection*{Si $bc|ac$ entonces $a|c$}

        % ======== DEMOSTRACION ========
        \begin{SmallIndentation}[1em]
            \textbf{Demostración}:

            Si $c=0$ esto se reduce a $0|0$ lo cual es cierto.

            Si $bc|ac$ entonces $ac = q(bc)$, por lo tanto ya que estamos en los
            enteros podemos cancelar y ver que $a = bq$ es decir $b|a$.

        \end{SmallIndentation}

    \subsection{Problema 11}
    \subsection*{Nunca se cumple que $4|n^2 +2$}

        % ======== DEMOSTRACION ========
        \begin{SmallIndentation}[1em]
            \textbf{Demostración}:

            Suponga que $n$ es par, por lo tanto tenemos que:
            $(2k)^2+2$ se puede expresar como $4k^2 + 2$, por lo tanto no es divisible entre
            cuatro.

            Si $n$ es impar, tenemos que $(2k+1)^+2$ se puede expresar como $4k^2 + 4k + 1 + 2$
            es decir $4(k^2+k) +3$, por lo tanto tampoco es divisible entre cuatro.

        \end{SmallIndentation}


    \subsection{Problema 13}
    \subsection*{Si $k$ es primo entonces el producto de todos los primos menores o
        iguales que $k$ divide a $n^k - n$}

        % ======== DEMOSTRACION ========
        \begin{SmallIndentation}[1em]
            \textbf{Demostración}:

            Suponga que $k$ es primo, por lo tanto si $k=2$ tenemos que 
            $(n^2-n) = n(n-1)$ y como son dos números consecutivos mínimo uno
            es par. Por lo tanto $2|n^2 -n$.

            Si $k \neq 2$ entonces $k$ es impar, veamos que pasa con los primos
            primos:

            Si $k=3$ entonces $(n^3-n)=n(n+1)(n-1)$ es decir 3 números consecutivos
            por lo tanto es divisible entre 3 y también entre 2. Por lo tanto 
            $2*3|n^3-n$.

            Si $k=5$ entonces:
            \begin{MultiLineEquation*}{3}
                (n^5-n) 
                    &= n(n^4-1)                     \\
                    &= n(n^2+1)(n^2-1)              \\
                    &= n(n+1)(n-1)(n^2+1)           \\
                    &= n(n+1)(n-1)(n-2)(n+2)+5
            \end{MultiLineEquation*}
            es decir son 5 números consecutivos por lo tanto es divisible entre
            5, 3 y también entre 2. Por lo tanto $2*3*5|n^5-n$.

            De forma más general como $k-1$ es par y tenemos la expresión $n(n^k-1)$
            donde podemos expandir este polinomio para tener k números consecutivos
            (esto se prueba usando inducción) es decir, será divisible entre el
            producto de los primos menores o iguales
            que $k$.


        \end{SmallIndentation}



    \subsection{Problema 15}
    \subsection*{Si $x, y$ son impares entonces $(x^2 +y^2)$ es par
        pero no divisible entre 4}

        % ======== DEMOSTRACION ========
        \begin{SmallIndentation}[1em]
            \textbf{Demostración}:

            Pongamos que: $x = 2k_1 + 1$ y $x = 2k_2 + 1$, entonces:
            \begin{MultiLineEquation*}{3}
                x^2 + y^2
                    &= (2k_1 + 1)^2 + (2k_2 + 1)^2              \\
                    &= 4k_1^2 + 4k_1 + 1 + 4k_2^2 + 4k_2 + 1    \\
                    &= 4k_1^2 + 4k_1 + 4k_2^2 + 4k_2 + 2        \\
                    &= 4(k_1^2 + k_1 + k_2^2 + k_2) + 2         \\
                    &= 2(2(k_1^2 + k_1 + k_2^2 + k_2) + 1)
            \end{MultiLineEquation*}
                
            Gracias a la última línea vemos que que $x^2 + y^2$ es par, y gracias
            a la penúltima línea es vemos que no puede ser divisible entre 4
                

        \end{SmallIndentation}

    \subsection{Problema 17}
    \subsection*{$GCD(n, n+1) = 1$}

        % ======== DEMOSTRACION ========
        \begin{SmallIndentation}[1em]
            \textbf{Demostración}:

            Sea $d = GCD(n, n+1)$, ahora tenemos que $d|n$ y $d|n+1$, por lo tanto
            divide a cualquier combinación lineal como por ejempo $d|(-1)n +1(n+1)$
            entonces $d|1$ por lo tanto solo le queda a $d$ ser uno.
        \end{SmallIndentation}

    \subsection{Problema 17.1}
    \subsection*{$LCM(n, n+1) = |n(n+1)|$}

        % ======== DEMOSTRACION ========
        \begin{SmallIndentation}[1em]
            \textbf{Demostración}:

            Ya sabemos que $GCD(n, n+1) = 1$ por lo tanto 
            $(1)LCM(n, n+1) = |n(n+1)|$
        \end{SmallIndentation}


    \subsection{Problema 19}
    \subsection*{Cualquier conjunto de números primos a pares, son primos relativos}


        % ======== DEMOSTRACION ========
        \begin{SmallIndentation}[1em]
            \textbf{Demostración}:

            Por contradicción, supón que hay un conjunto donde no son primos 
            relativos, pero si sus pares de elementos son coprimos.

            Sabemos que:
            \begin{equation*}
                A = \Set{a_1, a_2, a_3,\dots, a_{n-1}, a_n}
                \text{ donde } 
                (a_i,a_j) = 1 \; \forall i,j, \MiniSpace i \neq j
            \end{equation*}

            Si el conjunto no fuera coprimo entonces pasaría que:
            $GCD(a_1, a_2, a_3,\dots, a_{n-1}, a_n) = d$ con $d \neq 1$

            Y por definición sabemos que $d|a_i \; \forall a_i \in S$

            Pero si para todos los pares de números tenemos que el único número que 
            divide a ambos es el uno.

            Así, ningún miembro de $A$ tiene un divisor común con $d$ lo que sea
            una contradicción.

            Por lo tanto, el conjunto de enteros que son relativamente primos
            en pares es también relativamente primo.

        \end{SmallIndentation}

    \clearpage
    \subsection{Problema 21}
    \subsection*{Demuestre que cualquier entero de la forma $6k+5$ es de la forma $3k-1$
        pero no de manera inversa}


        % ======== DEMOSTRACION ========
        \begin{SmallIndentation}[1em]
            \textbf{Demostración}:

            Si tenemos un número de la forma $6k+5$ entonces ve que $6k+5 = 3(2k+2)-1$

            Pero veamos una contraprueba para su inversa:
            Dado un número de la forma $3k-1$, por ejemplo 3, tenemos el $3(3)-1=8$ no lo
            podemos escribir de la forma $6k+5$, pues implica $6k+5=8$ es decir
            $6k=3$, lo cual obviamente no tiene solución en los enteros, por lo tanto
            queda demostrado que su inversa no es correcta.
            
        \end{SmallIndentation}



    \subsection{Problema 23}
    \subsection*{$n^2 = 3k$ ó $n^2 = 3k + 1$}


        % ======== DEMOSTRACION ========
        \begin{SmallIndentation}[1em]
            \textbf{Demostración}:

            Antes que nada recuerda que un cuadrado perfecto, lo podemos expresar como:
            \begin{itemize}
                \item $(3k+0)^2 = 9k^2 = 3(3k^2)$
                \item $(3k+1)^2 = 9k^2 + 6k + 1 = 3(3k^2 + 2k) + 1$
                \item $(3k+2)^2 = 9k^2 + 12k + 3 +1 = 3(3k^2 + 4k + 1) + 1$
            \end{itemize}

            Es decir, todo cuadrado perfecto o es divisible entre 3 o es de la forma
            $3k+1$.
            
        \end{SmallIndentation}



    \subsection{Problema 25}
    \subsection*{Demuestre que existe una cantidad infinita de enteros
        x, y tal que $x+y=100$ y $(x,y)=5$}


        % ======== DEMOSTRACION ========
        \begin{SmallIndentation}[1em]
            \textbf{Demostración}:

            Ve que una solución es $55 + 45 = 5$ y $(55, 45)=5$
            Para encontrar todas las demás soluciones simplemente 
            tenemos que:
            \begin{itemize}
                \item x = 55 + r
                \item y = 45 - r
            \end{itemize}

            donde $r = 100k$ y $k$ es cualquier entero tal que $(k, 55) = 1$
        \end{SmallIndentation}

    \clearpage

    \subsection{Problema 27}
    \subsection*{Encuentre los enteros que cumple con que $(a,b) = 10$ y $[a,b] = 100$}

        % ======== DEMOSTRACION ========
        \begin{SmallIndentation}[1em]
            \textbf{Demostración}:

            Empecemos por enunciar a más detalle las restricciones que nos estan dando
            como $(a, b) = 10$, entonces tenemos que 
            $10|a$ y $10|b$ por lo tanto $10 \leq a, b$.

            Y como $[a, b] = 100$ entonces $a|100$ y $b|100$, por lo tanto $a, b \leq 100$.

            Con lo cual sabemos que dichos enteros tiene que estar entre 10 y 100, ahora
            podemos ocupar que $(a, b) = 10$ y ver que tenemos que:

            $a = 10q_1$ y $b = 10q_2$ con $(q_1, q_2) = 1$.

            Ademas $(a,b)[a,b] = 1000$. Por lo tanto dichas parejas son:

            \begin{itemize}
                \item $(100, 10)$
                \item $(20, 50)$
            \end{itemize}

        \end{SmallIndentation}


    \subsection{Problema 29}
    \subsection*{$a, b \in \Integers$ existen enteros $x, y$ tal que
        $GCD(x, y) = b$  y $LCM(x, y) = a$ si y solo si $b|a$}

        % ======== DEMOSTRACION ========
        \begin{SmallIndentation}[1em]
            \textbf{Demostración}:

            Probemos por doble condicional.

            Empecemos de ida:
            
            Dado $GCD(x, y) = b$ por lo tanto $b|x$ y dado $LCM(x, y) = a$ por lo tanto $x|a$
            y ya que la divisibilidad es transicitva tenemos por lo tanto que $b|a$.


            Ahora de regreso regreso:

            Si $b|a$, entonces $a = bq$.
            Podemos decir que $GCD(b, a) = GCD(b, bq) = b \cdot GCD(1, q)$.
            Podemos decir que $MCL(b, a) = MCL(b, bq) = bq = a$.

            Por lo tanto propongamos que $x=a$ y a $y=b$ entonces tenemos que 
            se cumple la propiedad.

        \end{SmallIndentation}


    \clearpage
    \subsection{Problema 31}
    \subsection*{$a - b |a^n - b^n$}

        % ======== DEMOSTRACION ========
        \begin{SmallIndentation}[1em]
            \textbf{Demostración}:

            Para que fuera cierto teniamos que encontrar
            $a^n - b^n = a-b(q)$ podemos proponer de manera completamente arbitraria
            que:
            \begin{MultiLineEquation*}{3}
                (a-b) \sum_{k=0}^{n-1} a^k b^{n-1-k}
                    &= (a-b) \sum_{k=0}^{n-1} a^k b^{n-1-k}         \\
                    &= \sum_{k=0}^{n-1} a^{k+1} b^{n-1-k} 
                       -
                       \sum_{k=0}^{n-1} a^k b^{n-k}                 \\
                    &= a^n + \sum_{k=1}^{n-1} a^{k} b^{n-k} 
                       -
                       \sum_{k=1}^{n-1} a^k b^{n-k} 
                       -b^n                                         \\
                    &= a^n - b^n                                    
            \end{MultiLineEquation*}

        \end{SmallIndentation}



    \subsection{Problema 31.1}
    \subsection*{$a - 1 |a^n -1$}

        % ======== DEMOSTRACION ========
        \begin{SmallIndentation}[1em]
            \textbf{Ideas}:

            Es muy obvio esto si $n=1$, pues $a-1|a-1$ y con $n=2$, pues $a-1|a^2-1$
            ya que gracias a la diferencia de cuadrados tenemos que:
            $a-1|(a+1)(a-1)$.

            Con una $n$ par es también muy fácil pues basta con ver que podemos siempre
            factorizar un $a-1$, pero también podemos hacer lo mismo con un $n$ impar, 
            basta con ver la descomposición del polinomio.\\


            \textbf{Demostracion}:

            Recuerdas la serie geométrica, sino no te preocupes, pues tenemos que:
            \begin{MultiLineEquation*}{3}
                a + ar + ar^2 + ar^3 + \cdots + ar^{n-1}
                    &= \sum_{k=0}^{n-1} ar^k
                    &= a\dfrac{1-r^n}{1-r}
                    &= a\dfrac{(-1)(r^n - 1)}{(-1)r-1}
                    &= a\dfrac{r^n - 1}{r-1}
            \end{MultiLineEquation*}

            Por lo tanto si pones a $a=1$ y $r=a$ tienes que:
            \begin{MultiLineEquation*}{3}
                1 + a + a^2 + a^3 + \cdots + a^{n-1}
                    &= \sum_{k=0}^{n-1} a^k
                    &= \dfrac{1-a^n}{1-a}
                    &= \dfrac{(-1)(a^n - 1)}{(-1)a-1}
                    &= \dfrac{a^n - 1}{a-1}
            \end{MultiLineEquation*}

            Por lo tanto ya que solo estamos sumando enteros o potencias de enteros 
            $\dfrac{a^n - 1}{a-1}$ debe ser un entero, es decir $a - 1 |a^n -1$.

        \end{SmallIndentation}


    \clearpage
    \subsection{Problema 33}
    \subsection*{GCD(a,b,c) = GCD((a, b), c)}

        % ======== DEMOSTRACION ========
        \begin{SmallIndentation}[1em]
            \textbf{Demostración}:

            Usando la factorización de primos tenemos que:
            \begin{itemize}
                \item $a = \prod_i p^{\alpha_i}$
                \item $b = \prod_i p^{\beta_i}$
                \item $c = \prod_i p^{\gamma_i}$
            \end{itemize}

            Entonces tenemos que:
            \begin{MultiLineEquation*}{2}
                GCD(a,b,c)
                    &= \prod_i p^{min(\alpha_i, \beta_i, \gamma_i)}
                    &= \prod_i p^{min(min(\alpha_i, \beta_i), \gamma_i))}
                    &= GCD((a,b),c)
            \end{MultiLineEquation*}

        \end{SmallIndentation}

    \subsection{Problema 35}
    \subsection*{Si $GCD(b, c) = 1$ y $r|b$ entonces $GCD(r, c) = 1$}

        % ======== DEMOSTRACION ========
        \begin{SmallIndentation}[1em]
            \textbf{Demostración}:

            Usando la Identidad de Bezut esto esta regalado pues tenemos que
            $bx + cy = 1$ y $b = rq$ entonces 
            $r(qx) + cy = 1$ por lo tanto $GCD(r, c) = 1$. Que facil son ciertas
            demostraciones. 

        \end{SmallIndentation}





% ======================================================================================
% ==================================     PRIMOS    =====================================
% ======================================================================================
\clearpage
\section{Primos}

    \subsection{Problema 2}
    \subsection*{Un número $n \in \Integers$ es divisible entre 3 si y solo si
        la suma de digitos (en base 10) de $n$ es divisible entre 3}

        % ======== DEMOSTRACION ========
        \begin{SmallIndentation}[1em]
            \textbf{Demostración}:

            Antes que nada, recuerda que a $n$ lo puedes escribir como
            $n = a_0(10^0) + a_1(10^1) + a_2(10^2) + \dots + a_k(10^k)$.

            Ahora, también recuerda que $10 \equiv 1 \pmod{3}$.

            Ahora $3|n$ si y solo si $n \equiv 0 \pmod{3}$ y recuerda
            que podemos poner a $n$ escrito de otra forma:
            $a_0(10^0) + a_1(10^1) + a_2(10^2) + \dots + a_k(10^k) \equiv 0 \pmod{3}$
            y como recuerdas ($10 \equiv 1 \pmod{3}$) tenemos que esto ocurre
            si y solo si:  $a_0 + a_1 +\dots +a_k \equiv 0 \pmod{3}$, esto es lo mismo que
            $3|a_0+a_1+a_2+\dots$.

            Es decir, un número $n \in \Integers$ es divisble entre 3 si y solo si
            la suma de digitos de $n$ es divisible entre 3.

        \end{SmallIndentation}



    \subsection{Problema 3}
    \subsection*{Cualquier número es divisible entre 11 si y solo si la diferencia de la suma
        de los dígitos impares y los dígitos pares son divisibles entre 11}

        % ======== DEMOSTRACION ========
        \begin{SmallIndentation}[1em]
            \textbf{Demostración}:

            Antes que nada, recuerda que a $n$ lo puedes escribir como
            $n = a_0(10^0) + a_1(10^1) + a_2(10^2) + \dots + a_k(10^k)$.

            Ahora, veamos este curioso patrón donde esta la clave:
            \begin{itemize}
                 \item $10 \equiv -1 \pmod{11}$
                 \item $100 \equiv (10)(10) \equiv (-1)(-1) \equiv 1 \pmod{11}$
                 \item $1000 \equiv (100)(10)(10) \equiv (-1)(-1)(-1) \equiv -1 \pmod{11}$
                 \item $\dots$
             \end{itemize} 

            Por lo tanto vemos que que de manera general $10^n \equiv (-1)^n \pmod{11}$

            Entonces si un número $x$ es divisible entre 11 tendremos que $x \equiv 0 \pmod{11}$
            Por lo tanto $(1)a_0 + (10)a_1 + \dots +(10^k)a_k \equiv 0 \pmod{11}$
            es decir $(1)a_0 + (-1)a_1 + (1)a_1 +\dots +(-1)^{k-1}a_k \equiv 0 \pmod{11}$

            Que si te das cuenta, es lo que queriamos demostrar :D

        \end{SmallIndentation}


    \subsection{Problema 8}
    \subsection*{Un primo de la forma $3k+1$ es de la forma $6k+1$}

        % ======== DEMOSTRACION ========
        \begin{SmallIndentation}[1em]
            \textbf{Demostración}:

            Sabemos que $p = 3k-1$ por lo tanto $p-1 = 3k$ es decir $p-1$ es divisible entre
            3, por lo tanto $p-1 = 6k$.
            ¿Porque?

            Porque supongamos que $p-1=3k_0$ pero no $p-1=6k_1$ (osea $p-1=3(2k_1)$), por lo
            tanto tendrá que ser de la forma $p-1=3(2k_1+1)$ es decir impar, pero eso
            implicaría que $p$ sea par. Cosa que no puede ser.

            Así $p-1$ si es de la forma $p-1=6k$ por lo tanto $p = 6k+1$. 

        \end{SmallIndentation}


        \subsection{Problema 10}
        \subsection*{Si $x, y$ son impares entonces $x^2 + y^2$ no puede ser un cuadrado
        perfecto}

            % ======== DEMOSTRACION ========
            \begin{SmallIndentation}[1em]
                \textbf{Demostración}:

                Esta demostración se deduce de manera inmediata del siguiente problema, pero ya 
                que lo estoy haciendo en \LaTeX es tal fácil como un copy paste :D

                Antes que nada recuerda que un cuadrado perfecto, lo podemos expresar como:
                \begin{itemize}
                    \item $(3k+0)^2 = 9k^2 = 3(3k^2)$
                    \item $(3k+1)^2 = 9k^2 + 6k + 1 = 3(3k^2 + 2k) + 1$
                    \item $(3k+2)^2 = 9k^2 + 12k + 3 +1 = 3(3k^2 + 4k + 1) + 1$
                \end{itemize}

                Es decir, todo cuadrado perfecto o es divisible entre 3 o es de la forma
                $3k+1$.

                Dado esto tenemos que:
                \begin{MultiLineEquation*}{3}
                    (3k_1+1)^2 + (3k_2+1)^2
                        &= 9k_1^2 + 6k_1 + 1  +  9k_2^2 + 6k_2 + 1      \\
                        &= 9k_1^2 + 6k_1 + 9k_2^2 + 6k_2 + 2            \\
                        &= 9k_1^2 + 6k_1 + 9k_2^2 + 6k_2 + 2            \\
                        &= 3(3k_1^2 + 2k_1 + 3k_2^2 + 2k_2) + 2            
                \end{MultiLineEquation*}

                Por lo tanto no puede ser un cuadrado perfecto.

            \end{SmallIndentation}


    \clearpage
    \subsection{Problema 11}
    \subsection*{Si $x, y$ son coprimos con 3 entonces $x^2 + y^2$ no puede ser un cuadrado
        perfecto}

        % ======== DEMOSTRACION ========
        \begin{SmallIndentation}[1em]
            \textbf{Demostración}:

            Antes que nada recuerda que un cuadrado perfecto, lo podemos expresar como:
            \begin{itemize}
                \item $(3k+0)^2 = 9k^2 = 3(3k^2)$
                \item $(3k+1)^2 = 9k^2 + 6k + 1 = 3(3k^2 + 2k) + 1$
                \item $(3k+2)^2 = 9k^2 + 12k + 3 +1 = 3(3k^2 + 4k + 1) + 1$
            \end{itemize}

            Es decir, todo cuadrado perfecto o es divisible entre 3 o es de la forma
            $3k+1$.

            Veamos los casos posibles:
            \begin{itemize}
                \item $x=3k_1+1$ y $y=3k_2+1$

                    Dado esto tenemos que:
                    \begin{MultiLineEquation*}{3}
                        (3k_1+1)^2 + (3k_2+1)^2
                            &= 9k_1^2 + 6k_1 + 1  +  9k_2^2 + 6k_2 + 1      \\
                            &= 9k_1^2 + 6k_1 + 9k_2^2 + 6k_2 + 2            \\
                            &= 9k_1^2 + 6k_1 + 9k_2^2 + 6k_2 + 2            \\
                            &= 3(3k_1^2 + 2k_1 + 3k_2^2 + 2k_2) + 2            
                    \end{MultiLineEquation*}

                    Por lo tanto no puede ser un cuadrado perfecto.
                        
                \item $x=3k_1+1$ y $y=3k_2+2$
                    
                    Dado esto tenemos que:
                    \begin{MultiLineEquation*}{3}
                        (3k_1+1)^2 + (3k_2+2)^2
                            &= 9k_1^2 + 6k_1 + 1 + 9k_2^2 + 12k_2 +3+1      \\
                            &= 9k_1^2 + 6k_1 + 3 + 9k_2^2 + 12k_2 + 2       \\
                            &= 3(3k_1^2 + 2k_1 + 1 + 3k_2^2 + 4k_2) + 2
                    \end{MultiLineEquation*}

                    Por lo tanto no puede ser un cuadrado perfecto.


                \item $x=3k_1+2$ y $y=3k_2+2$

                    Dado esto tenemos que:
                    \begin{MultiLineEquation*}{3}
                        (3k_1+2)^2 + (3k_2+2)^2
                            &= 9k_1^2 + 12k_1 + 3+1  +  9k_2^2 + 12k_2 +3+1 \\
                            &= 9k_1^2 + 12k_1 + 6  +  9k_2^2 + 12k_2 + 2    \\
                            &= 3(3k_1^2 + 6k_1 + 2 + 3k_2^2 + 6k_2) + 2    
                    \end{MultiLineEquation*}

                    Por lo tanto no puede ser un cuadrado perfecto.
            \end{itemize}


        \end{SmallIndentation}









    \subsection{Problema 24.1}
    \subsection*{Si $a,b$ son naturales diferentes entonces
    $(a,c) = (a,b)$ implica que $[a,b] \neq [a,c]$}

    % ======== DEMOSTRACION ========
    \begin{SmallIndentation}[1em]
        \textbf{Demostración}:
        
        Ahora, si fuera el caso tendriamos que:
        $(a,b) \; [a,b] = ab$ y $(a,c) \; [a,c] = ac$
        por lo tanto $ac=ab$ pero dijimos que $b \neq c$.
        Contradicción.
    
    \end{SmallIndentation}


    \subsection{Problema 24.4}
    \subsection*{Si $p$ es un primo y $p|a$ además que $p|(a^2+b^2)$
        implica que $p|b$}

    % ======== DEMOSTRACION ========
    \begin{SmallIndentation}[1em]
        \textbf{Demostración}:
        
        Sea $p|a$ por lo tanto $p|a^2$ y podemos decir que $p$ divide
        a cualquier combinación lineal, por ejemplo $p|(-1)(a^2)+(1)(a^2+b^2)$
        por lo tanto $p|b^2$ pero como sabemos que $p$ es un primo la unica
        forma de que divida a $b$. Por lo tanto $p|b$.

    \end{SmallIndentation}


    \subsection{Problema 24.5}
    \subsection*{Si $p$ es un primo y $p|a^n$ entonces $p|a$}

    % ======== DEMOSTRACION ========
    \begin{SmallIndentation}[1em]
        \textbf{Demostración}:
        
        Tenemos la factorización prima de $a$, el hecho de elevar
        $a$ al cuadrado no crea o elimina factores primos, por lo tanto
        $p|a$ si y solo si $p|a^n$.

    \end{SmallIndentation}


    \subsection{Problema 24.9}
    \subsection*{Si $p$ es un primo y $p|(a^2+b^2)$
        y $p|(b^2+c^2)$ entonces $p|(a^2-c^2)$}

    % ======== DEMOSTRACION ========
    \begin{SmallIndentation}[1em]
        \textbf{Demostración}:
        
        Esto sale en un paso, pues $p$ divide a cualquier combinación lineal
        de sus multiplos, es decir $p|(a^2+b^2)-(b^2+c^2)$, es decir
        $p|a^2-c^2$

    \end{SmallIndentation}


    \subsection{Problema 24.11}
    \subsection*{Si $(a,b)=1$ entonces $(a^2, ab, b^2)$}

    % ======== DEMOSTRACION ========
    \begin{SmallIndentation}[1em]
        \textbf{Demostración}:
        
        Si $(a,b)=1$ no comparten primos en común.

        Recuerda:

        Podemos también podemos decir que dados dos enteros escritos en 
        su factorización prima tenemos que:
        \begin{itemize}
            \item $a = p_1^{e_1} p_2^{e_2} \dots p_k^{e_k}$
            \item $b = p_1^{f_1} p_2^{f_2} \dots p_k^{f_k}$
        \end{itemize}

        Entonces podemos definir al máximo común divisor como:
        \begin{equation}
            GCD(a, b) = p_1^{min(e_1, f_1)} p_2^{min(e_2, f_2)} \dots p_k^{min(e_k, f_k)}
        \end{equation}

        Por lo tanto $a^2, b^2, ab$ ninguno creará primos en común
        que no existian en $a, b$. Por lo tanto $(a^2, ab, b^2)$

    \end{SmallIndentation}

    \subsection{Problema 24.13}
    \subsection*{Contraejemplo: Si $b|(a^2+1)$ entonces $b|(a^4+1)$}

    % ======== DEMOSTRACION ========
    \begin{SmallIndentation}[1em]
        \textbf{Demostración}:
        
        Suponga $b=5$ y $a=2$, entonces $5|(4+1)$ pero no es cierto que
        $5|(16+1)$.
    \end{SmallIndentation}



    \subsection{Problema 24.15}
    \subsection*{$(a,b,c) = ((a,b),(a,c))$}

    % ======== DEMOSTRACION ========
    \begin{SmallIndentation}[1em]
        \textbf{Demostración}:
        
        Podemos también podemos decir que dados dos enteros escritos en 
        su factorización prima tenemos que:
        \begin{itemize}
            \item $a = p_1^{e_1} p_2^{e_2} \dots p_k^{e_k}$
            \item $b = p_1^{f_1} p_2^{f_2} \dots p_k^{f_k}$
            \item $c = p_1^{g_1} p_2^{g_2} \dots p_k^{g_k}$
        \end{itemize}

        Por lo tanto:
        \begin{MultiLineEquation*}{3}
            GCD(a, b, c) 
                &= p_1^{min(e_1, f_1, g_1)} \dots p_k^{min(e_k, f_k, g_k)}      \\
                &= p_1^{min(min(e_1, f_1), min(e_1,g_1))} \dots p_k^{min(min(e_k, f_k), min(e_k,g_k))}            \\
                &= GCD(GCD(a,b), GCD(b,c))
        \end{MultiLineEquation*}
            
    \end{SmallIndentation}
        









    \subsection{Problema 25}
    \subsection*{¿Para que enteros $\dfrac{n(n+1)}{2} | n! $}

        % ======== DEMOSTRACION ========
        \begin{SmallIndentation}[1em]
            \textbf{Demostración}:

            $\dfrac{n(n+1)}{2} | n!$ si y solo si $\dfrac{n!}{\dfrac{n(n+1)}{2}}$
            pertenece a los enteros.

            Por lo tanto $\dfrac{(n-1)!}{\dfrac{n+1}{2}}$ es decir
            si $\dfrac{2(n-1)!}{n+1}$.

            Eso no pertenece a los enteros si $n$ fuera par, pues $2(n-1)!$ son elementos
            pares y $n+1$ (al $n$ ser par) es impar, por lo tanto es imposible encontrar
            factores para cancelarlo.

            Si $n$ fuera impar entonces el tenemos que la expansión de terminos de
            $2(n-1)!$ tiene un número par de factores, veamos el termino de enmedio más
            uno (por ejemplo el 3 en $(1)(2)(3)(4)$), aquel número en esa posción en la 
            expansión $(n-1)!$ tendrá un valor de $\dfrac{n+1}{2}$ pero al multiplicarlo
            por dos tenemos justo un $n+1$ por lo tanto se cancela y dicho número será
            un entero.

            $\dfrac{n(n+1)}{2} | n! $ si y solo si $n$ es impar.

            Quiza el último párrafo no quedará tan claro, así que unos ejemplos aquí:

            \begin{itemize}
                \item 
                    $\dfrac{2(5-1)!}{6} 
                        = \dfrac{2[(1)(2)(3)(4)]}{6} 
                        = \dfrac{(1)(2)(6)(4)}{6}
                        = (1)(2)(4)$

                \item 
                    $\dfrac{2(9-1)!}{10} 
                        = \dfrac{2[(1)(2)(3)(4)(5)(6)(7)(8)]}{10} 
                        = \dfrac{(1)(2)(3)(4)(10)(6)(7)(8)}{10} 
                        = (1)(2)(3)(4)(6)(7)(8)$

                \item 
                    $\dfrac{2(7-1)!}{8} 
                        = \dfrac{2[(1)(2)(3)(4)(5)(6)]}{8} 
                        = \dfrac{(1)(2)(3)(8)(5)(6)}{8} 
                        = (1)(2)(3)(5)(6)$ 
            \end{itemize}

        \end{SmallIndentation}


    \clearpage
    \subsection{Problema 28}
    \subsection*{Todo número compuesto $n$ tiene un divisor $a$ tal que $a \leq \sqrt{n}$}

        % ======== DEMOSTRACION ========
        \begin{SmallIndentation}[1em]

            Dado un entero particular, ¿Cómo podemos saber si es primo o no?

            Si el número es compuesto, ¿Cómo podemos encontrar un divisor no trivial?

            La primera idea es verificar si todos los enteros menores son
            divisores, si los únicos divisores son el 1 y el -1 entonces
            el número será primo.

            Este método es simple pero costoso en términos de cómputo. Sin
            embargo la propiedad de arriba nos podría facilitar el cálculo.\\

            \textbf{Demostración}:

            En efecto, como n es compuesto, $n = ab$.

            Si $a = b$, es decir si es un cuadrado perfecto entonces
            $a = b = a^2 = \sqrt{n}$.

            En caso contrario podemos suponer, que $a<b$, si multiplicamos
            por $a$ tenemos que $a^2<ab$. Por lo tanto $a^2 < n$.
            Por lo que $a < \sqrt{n}$.

        \end{SmallIndentation}












\end{document}