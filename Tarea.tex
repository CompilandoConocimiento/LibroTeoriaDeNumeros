% ****************************************************************************************
% ************************      TAREA DE TEORIA DE NUMEROS   *****************************
% ****************************************************************************************

% =======================================================
% =======         HEADER FOR DOCUMENT        ============
% =======================================================
    % *********   DOCUMENT ITSELF   **************
    \documentclass[12pt, fleqn]{article}                             %Type of docuemtn and size of font and left eq
    \usepackage[margin=1.2in]{geometry}                             %Margins and Geometry pacakge
    \usepackage{ifthen}                                             %Allow simple programming
    \usepackage{hyperref}                                           %Create MetaData for a PDF and LINKS!
    \usepackage{pdfpages}                                           %Create MetaData for a PDF and LINKS!
    \hypersetup{pageanchor=false}                                   %Solve 'double page 1' warnings in build
    \setlength{\parindent}{0pt}                                     %Eliminate ugly indentation
    \author{Oscar Andrés Rosas}                                     %Who I am

    % *********   LANGUAJE AND UFT-8   *********
    \usepackage[spanish]{babel}                                     %Please use spanish
    \usepackage[utf8]{inputenc}                                     %Please use spanish - UFT
    \usepackage[T1]{fontenc}                                        %Please use spanish
    \usepackage{textcmds}                                           %Allow us to use quoutes
    \usepackage{changepage}                                         %Allow us to use identate paragraphs
    \usepackage{lipsum}                                             %Allow to put dummy text

    % *********   MATH AND HIS STYLE  *********
    \usepackage{ntheorem, amsmath, amssymb, amsfonts}               %All fucking math, I want all!
    \usepackage{mathrsfs, mathtools, empheq}                        %All fucking math, I want all!
    \usepackage{centernot}                                          %Allow me to negate a symbol
    \decimalpoint                                                   %Use decimal point

    % *********   GRAPHICS AND IMAGES *********
    \usepackage{graphicx}                                           %Allow to create graphics
    \usepackage{wrapfig}                                            %Allow to create images
    \graphicspath{ {Graphics/} }                                    %Where are the images :D

    % *********   LISTS AND TABLES ***********
    \usepackage{listings}                                           %We will be using code here
    \usepackage[inline]{enumitem}                                   %We will need to enumarate
    \usepackage{tasks}                                              %Horizontal lists
    \usepackage{longtable}                                          %Lets make tables awesome
    \usepackage{booktabs}                                           %Lets make tables awesome
    \usepackage{tabularx}                                           %Lets make tables awesome
    \usepackage{multirow}                                           %Lets make tables awesome
    \usepackage{multicol}                                           %Create multicolumns

    % *********   HEADERS AND FOOTERS ********
    \usepackage{fancyhdr}                                           %Lets make awesome headers/footers
    \pagestyle{fancy}                                               %Lets make awesome headers/footers
    \setlength{\headheight}{16pt}                                   %Top line
    \setlength{\parskip}{0.5em}                                     %Top line
    \renewcommand{\footrulewidth}{0.5pt}                            %Bottom line

    \lhead{                                                         %Left Header
        \hyperlink{section.\arabic{section}}                        %Make a link to the current chapter
        {\normalsize{\textsc{\nouppercase{\leftmark}}}}             %And fot it put the name
    }

    \rhead{                                                         %Right Header
        \hyperlink{section.\arabic{section}.\arabic{subsection}}    %Make a link to the current chapter
            {\footnotesize{\textsc{\nouppercase{\rightmark}}}}      %And fot it put the name
    }

    \rfoot{\textsc{\small{Tarea}}}                                  %This will always be a footer  

    \fancyfoot[L]{                                                  %Algoritm for a changing footer
        \ifthenelse{\isodd{\value{page}}}                           %IF ODD PAGE:
            {\href{https://compilandoconocimiento.com/yo/}          %DO THIS:
                {\footnotesize                                      %Send the page
                    {\textsc{Algebra Superior 2}}}}                 %Send the page
            {\href{https://compilandoconocimiento.com}              %ELSE DO THIS: 
                {\footnotesize                                      %Send the author
                    {\textsc{Facultad de Ciencias}}}}               %Send the author
    }
    
    
    
% ========================================
% ===========   COMMANDS    ==============
% ========================================

    % =====  GENERAL TEXT  ==========
    \newcommand \Quote {\qq}                                        %Use: \Quote to use quotes
    \newcommand \Over {\overline}                                   %Use: \Bar to use just for short
    \newcommand \ForceNewLine {$\Space$\\}                          %Use it in theorems for example
    
    \newenvironment{Indentation}[1][0.75em]                         %Use: \begin{Inde...}[Num]...\end{Inde...}
    {\begin{adjustwidth}{#1}{}}                                     %If you dont put nothing i will use 0.75 em
    {\end{adjustwidth}}                                             %This indentate a paragraph
    \newenvironment{SmallIndentation}[1][0.75em]                    %Use: The same that we upper one, just 
    {\begin{adjustwidth}{#1}{}\begin{footnotesize}}                 %footnotesize size of letter by default
    {\end{footnotesize}\end{adjustwidth}}                           %that's it


    % =====  GENERAL MATH  ==========
    \DeclareMathOperator \Space {\quad}                             %Use: \Space for a cool mega space
    \DeclareMathOperator \MiniSpace {\;}                            %Use: \Space for a cool mini space
    \newcommand \Such {\MiniSpace|\MiniSpace}                       %Use: \Such like in sets
    \newcommand \Also {\MiniSpace \text{y} \MiniSpace}              %Use: \Also so it's look cool
    \newcommand \Remember[1]{\Space\text{\scriptsize{#1}}}          %Use: \Remember so it's look cool

    \newtheorem{Theorem}{Teorema}[section]                          %Use: \begin{Theorem}[Name]\label{Nombre}...
    \newtheorem{Corollary}{Colorario}[Theorem]                      %Use: \begin{Corollary}[Name]\label{Nombre}...
    \newtheorem{Lemma}[Theorem]{Lemma}                              %Use: \begin{Lemma}[Name]\label{Nombre}...
    \newtheorem{Definition}{Definición}[section]                    %Use: \begin{Definition}[Name]\label{Nombre}...

    \newcommand{\Set}[1]{\left\{ \MiniSpace #1 \MiniSpace \right\}} %Use: \Set {Info}
    \newcommand{\Brackets}[1]{\left[ #1 \right]}                    %Use: \Brackets {Info} 
    \newcommand{\Wrap}[1]{\left( #1 \right)}                        %Use: \Wrap {Info} 
    \newcommand{\pfrac}[2]{\Wrap{\dfrac{#1}{#2}}}                   %Use: Put fractions in parentesis

    \newenvironment{MultiLineEquation}[1]                           %Use: To create MultiLine equations
        {\begin{equation}\begin{alignedat}{#1}}                     %Use: \begin{Multi..}{Num. de Columnas}
        {\end{alignedat}\end{equation}}                             %And.. that's it!
    \newenvironment{MultiLineEquation*}[1]                          %Use: To create MultiLine equations
        {\begin{equation*}\begin{alignedat}{#1}}                    %Use: \begin{Multi..}{Num. de Columnas}
        {\end{alignedat}\end{equation*}}                            %And.. that's it!


    % =====  LOGIC  ==================
    \DeclareMathOperator \doublearrow {\leftrightarrow}             %Use: \doublearrow for a double arrow
    \newcommand \lequal {\MiniSpace \Leftrightarrow \MiniSpace}     %Use: \lequal for a double arrow
    \newcommand \linfire {\MiniSpace \Rightarrow \MiniSpace}        %Use: \lequal for a double arrow
    \newcommand \longto {\longrightarrow}                           %Use: \longto for a long arrow

    % =====  NUMBER THEORY  ==========
    \DeclareMathOperator \Naturals  {\mathbb{N}}                     %Use: \Naturals por Notation
    \DeclareMathOperator \Primes    {\mathbb{P}}                     %Use: \Naturals por Notation
    \DeclareMathOperator \Integers  {\mathbb{Z}}                     %Use: \Integers por Notation
    \DeclareMathOperator \Racionals {\mathbb{Q}}                     %Use: \Racionals por Notation
    \DeclareMathOperator \Reals     {\mathbb{R}}                     %Use: \Reals por Notation
    \DeclareMathOperator \Complexs  {\mathbb{C}}                     %Use: \Complex por Notation

    % === LINEAL ALGEBRA & VECTORS ===
    \DeclareMathOperator \LinealTransformation {\mathcal{T}}        %Use: \LinealTransformation for a cool T
    \newcommand{\Mag}[1]{\left| #1 \right|}                         %Use: \Mag {Info} 

    \newcommand{\pVector}[1]{                                       %Use: \pVector {Matrix Notation} use parentesis
        \ensuremath{\begin{pmatrix}#1\end{pmatrix}}                 %Example: \pVector{a\\b\\c} or \pVector{a&b&c} 
    }
    \newcommand{\lVector}[1]{                                       %Use: \lVector {Matrix Notation} use a abs 
        \ensuremath{\begin{vmatrix}#1\end{vmatrix}}                 %Example: \lVector{a\\b\\c} or \lVector{a&b&c} 
    }
    \newcommand{\bVector}[1]{                                       %Use: \bVector {Matrix Notation} use a brackets 
        \ensuremath{\begin{bmatrix}#1\end{bmatrix}}                 %Example: \bVector{a\\b\\c} or \bVector{a&b&c} 
    }
    \newcommand{\Vector}[1]{                                        %Use: \Vector {Matrix Notation} no parentesis
        \ensuremath{\begin{matrix}#1\end{matrix}}                   %Example: \Vector{a\\b\\c} or \Vector{a&b&c}
    }

    % MATRIX
    \makeatletter                                                   %Example: \begin{matrix}[cc|c]
    \renewcommand*\env@matrix[1][*\c@MaxMatrixCols c] {             %WTF! IS THIS
        \hskip -\arraycolsep                                        %WTF! IS THIS
        \let\@ifnextchar\new@ifnextchar                             %WTF! IS THIS
        \array{#1}                                                  %WTF! IS THIS
    }                                                               %WTF! IS THIS
    \makeatother                                                    %WTF! IS THIS

    % TRIGONOMETRIC FUNCTIONS
    \newcommand{\Cos}[1]{\cos\Wrap{#1}}                             %Simple wrappers
    \newcommand{\Sin}[1]{\sin\Wrap{#1}}                             %Simple wrappers

    % === COMPLEX ANALYSIS ===
    \newcommand \Cis[1]  {\Cos{#1} + i \Sin{#1}}                    %Use: \Cis for cos(x) + i sin(x)
    \newcommand \pCis[1] {\Wrap{\Cis{#1}}}                          %Use: \pCis for the same ut parantesis
    \newcommand \bCis[1] {\Brackets{\Cis{#1}}}                      %Use: \bCis for the same to Brackets

    % === CALCULUS ===
    \newcommand \MiniDerivate[1][x] {\dfrac{d}{d #1}}               %Use: \MiniDerivate for simple use
    \newcommand \Derivate[2]                                        %Complete Derivate -- [f(x)][x]
        {\dfrac{d \; #1}{d #2}}                                     %Use: \Partial for simple use
    
    \newcommand \MiniUpperDerivate[2]                               %Mini Derivate High Orden Derivate -- [x][1]
        {\dfrac{d^{#2}}{d#1^{#2}}}                                  %Mini Derivate High Orden Derivate
    \newcommand \UpperDerivate[3]                                   %Complete High Orden Derivate -- [f(x)][x][1]
        {\dfrac{d^{#3} \; #1}{d#2^{#3}}}                            %Use: \UpperDerivate for simple use
    
    \newcommand \MiniPartial[1][x] {\dfrac{\partial}{\partial #1}} %Use: \MiniDerivate for simple use
    \newcommand \Partial[2]                                        %Complete Derivate -- [f(x)][x]
        {\dfrac{\partial \; #1}{\partial #2}}                      %Use: \Partial for simple use
    
    \newcommand \MiniUpperPartial[2]                                %Mini Derivate High Orden Derivate -- [x][1] 
        {\dfrac{\partial^{#2}}{\partial #1^{#2}}}                   %Mini Derivate High Orden Derivate
    \newcommand \UpperPartial[3]                                    %Complete High Orden Derivate -- [f(x)][x][1]
        {\dfrac{\partial^{#3} \; #1}{\partial#2^{#3}}}              %Use: \UpperDerivate for simple use


    % =====  GENERAL COLOR  =========
    \definecolor{IndigoMD}{HTML}{3F51B5}                            %Use: Color :D
    \definecolor{DeepPurpleMD}{HTML}{673AB7}                        %Use: Color :D
    \definecolor{TealMD}{HTML}{009688}                              %Use: Color :D        
    \definecolor{BlueGrey800MD}{HTML}{37474F}                       %Use: Color :D
    \definecolor{BlueGrey100MD}{HTML}{CFD8DC}                       %Use: Color :D
    \definecolor{IndigoMD}{HTML}{3F51B5}                            %Use: Color :D
    \definecolor{Green100MD}{HTML}{DCEDC8}                          %Use: Color :D

    \newenvironment{ColorText}[1]{                                  %Use: \begin{ColorText}
        \leavevmode\color{#1}\ignorespaces}                         %That's is!


    % =====  CODE EDITOR =========
    \lstdefinestyle{CompilandoStyle} {                              %This is Code Style
        backgroundcolor=\color{BlueGrey800MD},                      %Background Color  
        basicstyle=\tiny\color{white},                              %Font color
        commentstyle=\color{BlueGrey100MD},                         %Comment color
        stringstyle=\color{TealMD},                                 %String color
        keywordstyle=\color{Green100MD},                            %keywords color
        numberstyle=\tiny\color{TealMD},                            %Size of a number
        frame=shadowbox,                                            %Adds a frame around the code
        breakatwhitespace=true,                                     %Style                       
        breaklines=true,                                            %Style                   
        keepspaces=true,                                            %Style                   
        numbers=left,                                               %Style                   
        numbersep=10pt,                                             %Style 
        xleftmargin=\parindent,                                     %Style 
        tabsize=4                                                   %Style 
    }
 
    \lstset{style=CompilandoStyle}                                  %Use this style

% =====================================================
% ============        COVER PAGE       ================
% =====================================================
\begin{document}
\begin{titlepage}

    \center
    % ============ UNIVERSITY NAME AND DATA =========
    \textsc{\Large Algebra Superior 2}\\[0.5cm] 
    \textsc{\large Grupo 4098}\\[1.0cm]

    % ============ NAME OF THE DOCUMENT  ============
    \rule{\linewidth}{0.5mm} \\[1.0cm]
        { \huge \bfseries Soluciones y Demostraciones}\\[1.0cm] 
    \rule{\linewidth}{0.5mm} \\[1.5cm]
     
    % ============  MY INFORMATION  =================
    \begin{minipage}{0.55\textwidth}
        \begin{flushleft}
            \footnotesize{
            \textbf{\textsc{Alumnos:}}\\
                \begin{itemize}
                    \item Palacios Rodríguez Ricardo Rubén
                    \item Rosas Hernandez Oscar Andres
                    \item José Martín Panting Magaña
                    \item Raúl Leyva Cedillo
                    \item Angel Mariano Guiño Flores
                    \item Gloria Guadalupe Cervantes Vidal
                    \item David Iván Morales Campos
                    \item Aaron Barrera Tellez
                    \item Elias Garcia Alejandro
                    \item Víctor Hugo García Hernández
                    \item Oscar Márquez Esquivel
                \end{itemize}
            }
        \end{flushleft}
    \end{minipage}
    ~
    \begin{minipage}{0.4\textwidth}
        \begin{flushright} \footnotesize
            \textbf{\textsc{Profesor: }}\\
            Leonardo Faustinos Morales
            
            \vspace{2em}

            \textbf{\textsc{Ayudante: }}\\
            Jonathan López Ruiz
        \end{flushright}
    




\end{minipage}\\[3,5cm]
    
    
    % ====== DATE ================
    {\large Lunes 30 de Octubre}\\[2cm] 

    \vfill

\end{titlepage}


% ======================================================================================
% ==================================     TAREA     =====================================
% ======================================================================================


% =====================================================
% ================     EJERCICIO 1   ==================
% =====================================================
\section{Ejercicio 1}

    \textbf{Muestra un sistema reducido de residuos modulo 7 compuesto solo de potencias en 3}

    % ======== DEMOSTRACION ========
    \begin{SmallIndentation}[1em]
        \textbf{Solución}:
        
        Un sistema reducido de residuos módulo 7 sencillo es $Residuos = \Set{0, 1, 2, 3, 4, 5, 6}$
        ahora, podemos decir que:

        \begin{itemize}
            \item $3^0 \equiv 1         \pmod{7}$
            \item $3^2 = 9 \equiv 2     \pmod{7}$
            \item $3^1 \equiv 3         \pmod{7}$
            \item $3^4 = 81 \equiv 4    \pmod{7}$
            \item $3^5 = 243 \equiv 5   \pmod{7}$
            \item $3^3 = 27 \equiv 6    \pmod{7}$
        \end{itemize}

        Pero... que pasa con el 0, es lo único que nos falta, pero resulta que es imposible 
        encontrar un número tal que $3^n = 7k$ pues $3^n$ esta formado solo por 3, esta es
        su factorización prima, por lo tanto será imposible que alguna vez este número sea
        también divisible entre 7, pues esto significaría que podemos encontrar un 7 en su
        factorización prima.
    
    \end{SmallIndentation}
        

% =====================================================
% ================     EJERCICIO 2   ==================
% =====================================================
\section{Ejercicio 2}

    \textbf{Probar que $n^{6k} -1 |7 \Space \forall k$}

    % ======== DEMOSTRACION ========
    \begin{SmallIndentation}[1em]
        \textbf{Solución}:
        
        Vamos a probar que $n^{6k} \equiv 1 \pmod{7}$

        Basta con ver que $n^6 \equiv 1 \pmod{7}$ porque si esto fuera cierto, entonces
        elevar ese uno a la $k$ seguira siendo uno.

        Ve $n^6 \equiv 1 \pmod{7}$ se puede demostrar muy facil con el Teorema de Fermat
        ya que $(n, 7) = 1$ (es decir, no pertenece a la clase de equivalencia del 0 módulo 7)
        entonces $n^{\phi(7)} \equiv 1 \pmod{7}$
    
    \end{SmallIndentation}


% =====================================================
% ================     EJERCICIO 3   ==================
% =====================================================
\section{Ejercicio 3}

    \textbf{Probar que $n^2 - a^2$ es divisible entre 91 si n y a son primos relativos con 91}

    % ======== DEMOSTRACION ========
    \begin{SmallIndentation}[1em]
        \textbf{Solución}:
        
        Supón que $n = 5$ pues $(91, 5) = 1$ y $(91, 5) = 1$
        y que $a = 3$ pues $(91, 3) = 1$ y $(91, 3) = 1$

        Pero $5^2- 3^2$ es 16 y 16 no es divisible entre 91 
    
    \end{SmallIndentation}

    \clearpage


    \textbf{Probar que $n^{12} - a{12}$ es divisible entre 91 si n y a son primos relativos con 91}

    % ======== DEMOSTRACION ========
    \begin{SmallIndentation}[1em]
        \textbf{Solución}:
        
        Como $n$ y $a$ son primos relativos con 91 (y $91 = 13*7$), también son primos relativos con 13 y con 7
        por lo tanto obtenemos que:
        \begin{itemize}
            \item $n^{\phi(13)} \equiv 1 \pmod{13}$ y $a^{\phi(13)} \equiv 1 \pmod{13}$
            \item $n^{\phi(7)} \equiv 1 \pmod{7}$ y $a^{\phi(7)} \equiv 1 \pmod{7}$
        \end{itemize}

        Con lo tanto tenemos que $n^{6} - a^{6} \equiv 0 \pmod{7}$ y si la elevamos al cuadrado tenemos
        que $n^{12} - a^{12} \equiv 0 \pmod{7}$ es decir $n^{12} - a^{12}$ es un multiplo de 12.

        Por otro lado $n^{12} - a^{12} \equiv 0 \pmod{13}$ es decir $n^{12} - a^{12}$ es un multiplo de 13.

        Y como es multiplo de ambos de sus factores, tenemos que $n^{12} - a^{12}$ es divisible entre 91
    
    \end{SmallIndentation}
        



% =====================================================
% ================     EJERCICIO 4   ==================
% =====================================================
\section{Ejercicio 4}

    \textbf{Cual es el último digito de $3^{400}$}

    % ======== DEMOSTRACION ========
    \begin{SmallIndentation}[1em]
        \textbf{Solución}:
        
        Si te das cuenta lo único que te están pidiendo es que $3^{400} \pmod{10}$
        ahora como $(3, 10) = 1$ entonces podemos aplicar la el Teorema de Fermat
        donde $a^{\phi(n)} \equiv 1 \pmod{n}$ es decir $3^{4} \equiv 1 \pmod{10}$
        por lo tanto también tenemos que $(3^{4})^{100} \equiv 1^{100} = 1 \pmod{10}$

        Por lo tanto el último digito es uno
    
    \end{SmallIndentation}



% =====================================================
% ================     EJERCICIO 5   ==================
% =====================================================
\section{Ejercicio 5}

    \textbf{Encontrar el número de enteros positivos $\leq$ 25200 que son primos
    relativos con 3600}

    % ======== DEMOSTRACION ========
    \begin{SmallIndentation}[1em]
        \textbf{Solución}:
        
        Ve que usando el Teorema del Ejercicio 6 (Demostrar que si m y k son enteros
        positivos, entonces el número de enteros positivos menos o iguales a mk que son
        primos relativos con m es $k\phi(m)$)

        Entonces tenemos que $m = 7$ y $k = 3600$, entonces $7 \phi(3600)$ son el número
        de enteros positivos menos o iguales a 25,200 que son primos relativos con 3600.

        Y:
        \begin{MultiLineEquation*}{3}
             7 \phi(3600)
                &= 7 \phi((2^4)(3^2)(5^2))  \\
                &= (7)(3600)\Wrap{1 - \dfrac{1}{2}}\Wrap{1 - \dfrac{1}{3}}\Wrap{1 - \dfrac{1}{5}}   \\
                &= (7)(3600)\pfrac{1}{2}\pfrac{2}{3}\pfrac{4}{5}                                    \\
                &= 6720
         \end{MultiLineEquation*}
              
    
    \end{SmallIndentation}



% =====================================================
% ================     EJERCICIO 6   ==================
% =====================================================
\section{Ejercicio 6}

    \textbf{$\phi(n)$ es la cantidad de naturales que son primos relativos menores que n, pero 
    también $n$ es la cantidad de primos relativos con $n$ en el segmento $[n+1, 2n]$}

        % ======== DEMOSTRACION ========
        \begin{SmallIndentation}[1em]
            \textbf{Demostración}:
            
            Esto puede ser muy obvio o como para mi extremadamente interesante, para verlo
            tenemos que recordar la clave... $GCD(a,b) = GCD(a, b+ak)$.

            Usando esto podemos darnos cuenta que cualquier entero $k$ que cumpla con que
            $GCD(n, k) = 1 = GCD(n, k+n)$ es decir, si $k$ es un coprimo con $n$ entonces
            $k+n$ también lo será, por lo tanto puedo hacer esto con cada uno y solo con cada
            uno de los elementos que contaba la $phi(n)$.

            Otra forma de decir lo que acabo de decir que podemos hacer una biyección entre ambos
            intervalos usando lo que acabo de decir, a cada elemento $a \in [1, n-1]$ tal que $(a, n) = 1$
            lo vamos a relacionar con $a+n$ que pertenece a $[n+1, 2n]$
        
        \end{SmallIndentation}
            

    \textbf{Para cuales quiera $a, b$ se cumple que el número de naturales menores o iguales que $ab$
    que son primos relativos con $b$ es $a \phi(b)$}

    % ======== DEMOSTRACION ========
    \begin{SmallIndentation}[1em]
        \textbf{Solución}:
        
        Sea $\phi(b)$ la cantidad de naturales que son coprimos con b y que estan en el intervalo
        $[1, b-1]$.

        Ya que sabemos que $\phi(n)$ es la cantidad de naturales que son primos relativos menores
        que n, pero también $n$ es la cantidad de primos relativos con $n$ en el
        segmento $[n+1, 2n]$... o incluso más general en el intervalo $[kn+1, (k+1)n]$.

        Por lo tanto sabemos que la cantidad de primos relativos con $b$ entre $[1, ab]$ es
        simplemente la suma de $a$ intervalos, y ya sabemos que la cantidad de primos relativos
        con b en cada intervalo es $\phi(b)$ por lo tanto $a\phi(b)$ nos dará la cantidada de
        primos relativos en el intervalo $[1, ab]$.
    
    \end{SmallIndentation}



% =====================================================
% ================     EJERCICIO 7   ==================
% =====================================================
\clearpage
\section{Ejercicio 7}

    \textbf{Para $n > 2$ se tiene que $\phi(n)$ es un número par}
                
        % ======== DEMOSTRACION ========
        \begin{SmallIndentation}[1em]
            \textbf{Demostración}:

            El caso $n = 2^k$ con $k > 1$ se sigue con facilidad

            Si $n$ no es una potencia de 2, entonces lo divide un
            primo impar p y entonces $n = p^\alpha m$ con $(p^\alpha, m) = 1$
            por lo que:
            $\phi(n) = \phi(p^\alpha)\phi(m)$ y como
            $\phi(p^\alpha) = p^{\alpha-1}(p-1)$ y $(p-1)$ es par, el resultado se sigue.

        \end{SmallIndentation}

    \textbf{Si n tiene k factores primos impares distintos, entonces $2^k|\phi(n)$}

    % ======== DEMOSTRACION ========
    \begin{SmallIndentation}[1em]
        \textbf{Solución}:
        
        Esta se ve fea, pero la verdad es que no lo esta, suponte que tenemos a $n$
        como un producto de $k$ primos mayores que dos elevados a una potencia.

        Ahora, como la Phi de Euler es una función multiplicativa (siempre que sean
        primos relativos, pero ya que vamos a trabajar con la factorización de $n$ creo que doy esto
        por obvio) podemos ver que:

        La $\phi(n)$ es el producto de las phis de cada uno de dichos factores impares,
        además recuerda que si $n < 2$ entonces $\phi(n)$ es par, ahí esta la clave.

        Gracias a lo anterior podemos separar la $\phi(n)$ en $k$ productos pares, donde $k$ es el número
        de factores primos impares, ahora de como son productos pares podemso factorizar un dos de cada uno 
        por lo tanto, al momento de calcular la $\phi(n)$ podremos factorizar $k$ veces el número 2,
        por lo tanto $2^k|\phi(n)$         
    
    \end{SmallIndentation}


% =====================================================
% ================     EJERCICIO 8   ==================
% =====================================================
\clearpage
\section{Ejercicio 8}

    \textbf{Calcula $\phi(35)$}

    % ======== DEMOSTRACION ========
    \begin{SmallIndentation}[1em]
        \textbf{Solución}:
        \begin{MultiLineEquation*}{3}
            \phi(35) 
                &= \phi(7*5) = 35 \Wrap{1 - \dfrac{1}{5}}\Wrap{1 - \dfrac{1}{7}}        \\
                &= 35 \pfrac{4}{5} \pfrac{6}{7}                                         \\
                &= 35 \pfrac{24}{35}                                                    \\
                &= 24
        \end{MultiLineEquation*}
    
    \end{SmallIndentation}


    \textbf{Calcula $\phi(105)$}

    % ======== DEMOSTRACION ========
    \begin{SmallIndentation}[1em]
        \textbf{Solución}:
        \begin{MultiLineEquation*}{3}
            \phi(105) 
                &= \phi(5*3*7)
                = 105 \Wrap{1 - \dfrac{1}{3}}\Wrap{1 - \dfrac{1}{5}}\Wrap{1 - \dfrac{1}{7}}     \\
                &= 105 \pfrac{2}{3} \pfrac{4}{5} \pfrac{6}{7}                                   \\
                &= 105 \pfrac{48}{105}                                                          \\
                &= 48
        \end{MultiLineEquation*}
    
    \end{SmallIndentation}

    \textbf{Calcula $\phi(333)$}

    % ======== DEMOSTRACION ========
    \begin{SmallIndentation}[1em]
        \textbf{Solución}:
        \begin{MultiLineEquation*}{3}
            \phi(333) 
                &= \phi(3^2 \cdot 37)
                 = 333 \Wrap{1 - \dfrac{1}{3}}\Wrap{1 - \dfrac{1}{37}}          \\
                &= 333 \pfrac{2}{3} \pfrac{36}{37}                              \\
                &= (3)(2)(36) = 216 
        \end{MultiLineEquation*}
    
    \end{SmallIndentation}


    \textbf{Calcula $\phi(2401)$}

    % ======== DEMOSTRACION ========
    \begin{SmallIndentation}[1em]
        \textbf{Solución}:
        \begin{MultiLineEquation*}{3}
            \phi(2401) 
                &= \phi(7^4)
                 = 2401 \Wrap{1 - \dfrac{1}{7}}                 \\
                &= 2401 \pfrac{6}{7}                            \\
                &= 7^3(6) = 2058 
        \end{MultiLineEquation*}
    
    \end{SmallIndentation}




% =====================================================
% ================     EJERCICIO 9   ==================
% =====================================================
\clearpage
\section{Ejercicio 9}

    Esta tarea no se debería leer en orden cronologico porque voy a usar un teorema de demuestro
    en el siguiente ejercicio, es fácil, veamos que ya sabemos que:

    Si $(a, n) = 1$ y $(a-1, n) = 1$ es decir si tanto $a$ como su antecesor es primo
    relativo con m tenemos que:
    $1 + a + a^2 + \dots + a^{\phi(n)-1} \equiv 0 \pmod{n}$

    Y ahora toma a $a = 9$ y $n = 35$, por lo tanto $\phi(35)-1 = 23$, por lo tanto:
    $1 + 9 + 9^2 + \dots + a^{23} \equiv 0 \pmod{35}$.

    Ahora si lo que quieres es comprobarlo simplemente hacemos la geometrica:
    \begin{MultiLineEquation*}{3}
        \sum_{k=0}^{23} 9^k 
            &= \dfrac{9^{23}-1}{8}                      \\
            &= \dfrac{7976644307687250986336}{8}        \\
            &= 35 \pfrac{2279041230767785996096}{8}     \\
            &= 35 (18232329846142287968768)
    \end{MultiLineEquation*}

% =====================================================
% ================     EJERCICIO 10  ==================
% =====================================================
\clearpage
\section{Ejercicio 10}

    \textbf{Si $(a, n) = 1$ y $(a-1, n) = 1$ es decir si tanto $a$ como su antecesor es primo
    relativo con m tenemos que}:

    $1 + a + a^2 + \dots + a^{\phi(n)-1} \equiv 0 \pmod{n}$

    % ======== DEMOSTRACION ========
    \begin{SmallIndentation}[1em]
        \textbf{Demostración}:
        
        Esta demostración pide a gritos una serie geométrica, hagamosla y veamos que:
        \begin{MultiLineEquation*}{3}
            1 + a + a^2 + \dots + a^{\phi(n)-1}
                &= \sum_{k=0}^{\phi(n)-1} a^k
                &= \dfrac{a^{\phi(n)}-1}{a-1}
        \end{MultiLineEquation*}

        Ahora, sabemos que como $(a, n) = 1$ entonces $a^{\phi(n)} \equiv 1 \pmod{n}$
        es decir $n | a^{\phi(n)} - 1$ es decir $a^{\phi(n)} - 1 = qn$ entonces 
        en vez de decir $\frac{a^{\phi(n)}-1}{a-1}$ podriamos decir $\frac{nq}{a-1}$
        
        Ahora recuerda que $\frac{nq}{a-1}$ es un entero, pero aun más que como $(a-1, n)=1$
        $a-1$ no elimina tienen ningún factor en común con $n$, por lo tanto podemos sacar comodamente
        a $n$ y decir que $1 + a + a^2 + \dots + a^{\phi(n)-1} = \frac{a^{\phi(n)}-1}{a-1} = n \frac{q}{a-1}$
        es decir $1 + a + a^2 + \dots + a^{\phi(n)-1}$ es un multiplo de $n$, por lo tanto
        es congruente con cero módulo n

    \end{SmallIndentation}
        



% =====================================================
% ================     EJERCICIO 11  ==================
% =====================================================
\clearpage
\section{Ejercicio 11}

    \textbf{Para cualquier entero n, tal que $(n,10) = 1$ tenemos que $n$ divide a algún N que consiste de
    solamente unos en su representación decimal}:

    % ======== DEMOSTRACION ========
    \begin{SmallIndentation}[1em]
        \textbf{Demostración}:
        
        Esta se ve difícil pero veras que solo es una aplicación del teorema:
        Si $(a, n) = 1$ y $(a-1, n) = 1$ es decir si tanto $a$ como su antecesor es primo
        relativo con m tenemos que:
        $1 + a + a^2 + \dots + a^{\phi(n)-1} \equiv 0 \pmod{n}$

        Si $(n,10)=1$ y $(n,9)=1$. Y con esto podemos aplicar ya el Teorema, pues podemos ver a 
        $N$ como $10^0 + 10^1 + 10^{\phi(n)-1}$.

        Gracias a ese Teorema vemos que esa N que son puros unos contendrá a $\phi(n)-1$ unos, quiza
        no sea la menor $N$ pero es una válida siempre.

        Pero... que pasa si $(9, n) \neq 1$.

        Entonces piensa que si $(n, 10) = 1$ si y solo si $(9n, 10)$ pues $9n=(n)(3)(3)$ y la única forma en 
        que esto afectaría el valor del gcd es que 10 compartierá algun factor con nueve, pero como no, podemos
        afirmar esto.

        Entonces $10^{\phi(9n) \equiv 1 \pmod(9n)}$ por el Teorema de Fermat-Euler si $(10, 9n)=1$ lo que implica
        que $(10, n)$.

        Ahora, ya que sabemos que $10^{\phi(9n) \equiv 1 \pmod(9n)}$ podemos escribirlo como que:
        $10^{\phi(9n)} - 1 = 9nq$ que podemos poner como:
        \begin{MultiLineEquation*}{3}
            nq 
                &= \dfrac{10^{\phi(9n)} - 1}{9}         \\  
                &= \dfrac{10^{\phi(9n)} - 1}{10 - 1}    \\
                &= \sum_{k=0}^{\phi(9n)-1} 10^k
        \end{MultiLineEquation*}

        Por lo tanto siempre lo podemos escribir como una suma geometríca, es decir una N que esta hecha
        de puros unos.

        Por lo tanto y como resumen:

        \begin{itemize}
            \item Si es que $(n, 10) = 1$ y $n \neq 3k$ entonces un número creado por $\phi(n)-1$ unos
                siempre será un múltiplo de $n$.

            \item Si es que $(n, 10) = 1$ y $n = 3k$ entonces un número creado por $\phi(9n)-1$ unos
                siempre será un múltiplo de $n$.
        \end{itemize}

    \end{SmallIndentation}


    % ======== DEMOSTRACION ========
    \clearpage
    \begin{SmallIndentation}[1em]
        \textbf{Demostración Versión 2}:
        
        Para alguna $n$ arbitaria crea el conjunto $Unos = {1, 11, 111, 1111, \dots \sum_{k=0}^{N}10^k }$.
        Ahora, nota que $Unos$ tiene $N+1$ elementos, desde $0$ hasta $N$.

        Ahora ve los residuos que obtiene al dividir cada elementos de $Unos$ entre $n$. Si te das cuenta
        al dividir entre $n$ solo tendrás $n$ residuos posibles, pero como tienes $N+1$ elementos 2 elementos
        diferentes tendrán un mismo residuo, por lo tanto su resta será divisible entre $n$ (llamemos a la resta
        $N'$).

        Entonces para cualquier $n$ tenemos que es posible encontrar una $N'$ que esta hecha de la resta de dos
        números que contienen puros unos en su representación decimal, por lo tanto $N'$ tendrá la forma de
        $N' = 1\cdots0\cdots$, es decir $N'$ esta formada de $q$ unos consecutivos y de $k$ ceros consecutivos.

        Finalmente si $(10, n) =1$ y sabemos que $n|N'$ tenemos que $n|(N)(10^k)$ con $N$ siendo un número de puros
        unos, por lo tanto tenemos que $n|N$ pues $n$ no puede dividir a $10^k$ pues es primo relativo con 10.

    \end{SmallIndentation}
        


\end{document}